\documentclass[12pt]{article}

\usepackage[utf8]{inputenc}
\usepackage{amsmath, amssymb, amsfonts}
\usepackage{geometry}
\usepackage{graphicx}
\usepackage{fancyhdr}

\geometry{a4paper, margin=1in, headheight=45pt, top=1.2in}

\pagestyle{fancy}
\fancyhf{}
\renewcommand{\headrulewidth}{0.4pt}

\fancyhead[L]{
    \includegraphics[height=1.5cm]{/home/keerthana-s/Pictures/lo.png}
}

\fancyhead[R]{
    \textbf{Name:} Keerthana S \\
    \textbf{ID:} COMETFWC047 \\
    \textbf{Date:} 3 feb 2026
}

\begin{document}

\section*{30th International Mathematical Olympiad (1989)}

\subsection*{Day I}

1. Prove that the set $\{1,2,\dots,1989\}$ can be expressed as the disjoint union of subsets
$A_i$ $(i=1,2,\dots,117)$ such that each $A_i$ contains 17 elements and the sum of the
elements of each $A_i$ is the same.

2. In an acute-angled triangle $ABC$, the internal bisector of angle $A$ meets the circumcircle
again at $A_1$. Points $B_1$ and $C_1$ are defined similarly. Let $A_0$ be the intersection of
$AA_1$ with the external bisectors of angles $B$ and $C$. Points $B_0$ and $C_0$ are defined
similarly. Prove that:
\begin{itemize}
\item[(i)] The area of triangle $A_0B_0C_0$ is twice the area of the hexagon
$AC_1BA_1CB_1$.
\item[(ii)] The area of triangle $A_0B_0C_0$ is at least four times the area of triangle $ABC$.
\end{itemize}

3. Let $n$ and $k$ be positive integers and let $S$ be a set of $n$ points in the plane such that
no three points are collinear. If for any point $P$ of $S$ there are at least $k$ points of $S$
equidistant from $P$, prove that
\[
k < \frac{1}{2} + \sqrt{2n}.
\]

\subsection*{Day II}

4. Let $ABCD$ be a convex quadrilateral such that $AB = AD + BC$. There exists a point $P$
inside the quadrilateral at a distance $h$ from the line $CD$ such that
$AP = h + AD$ and $BP = h + BC$. Show that
\[
\frac{1}{\sqrt{h}} \ge \frac{1}{\sqrt{AD}} + \frac{1}{\sqrt{BC}}.
\]

5. Prove that for each positive integer $n$ there exist $n$ consecutive positive integers none
of which is an integral power of a prime number.

6. A permutation $(x_1,x_2,\dots,x_{2n})$ of the set $\{1,2,\dots,2n\}$ has property $P$ if
\[
|x_i - x_{i+1}| = n
\]
for at least one $i$. Show that for each $n$ there are more permutations with property $P$
than without.

\newpage

\section*{31st International Mathematical Olympiad (1990)}

\subsection*{Day I}

1. Chords $AB$ and $CD$ of a circle intersect at a point $E$ inside the circle. Let $M$ be an
interior point of segment $EB$. The tangent at $E$ to the circle through $D,E,M$ meets the
lines $BC$ and $AC$ at $F$ and $G$ respectively. If
\[
\frac{AM}{AB} = t,
\]
find $\frac{EG}{EF}$ in terms of $t$.

2. Let $n \ge 3$ and let $E$ be a set of $2n-1$ distinct points on a circle. Suppose exactly
$k$ points are colored black. Such a coloring is called good if there exists a pair of black
points such that the interior of one arc between them contains exactly $n$ points of $E$.
Find the smallest value of $k$ such that every coloring is good.

3. Determine all integers $n > 1$ such that
\[
\frac{2^n + 1}{n^2}
\]
is an integer.

\subsection*{Day II}

4. Let $\mathbb{Q}^+$ be the set of positive rational numbers. Construct a function
$f : \mathbb{Q}^+ \to \mathbb{Q}^+$ such that
\[
f(xf(y)) = \frac{f(x)}{y}
\]
for all $x,y \in \mathbb{Q}^+$.

5. Starting from an integer $n_0 > 1$, two players $A$ and $B$ choose integers alternately.
Player $A$ chooses $n_{2k+1}$ with
\[
n_{2k} \le n_{2k+1} \le n_{2k}^2,
\]
and player $B$ chooses $n_{2k+2}$ such that
\[
\frac{n_{2k+1}}{n_{2k+2}}
\]
is a prime power. Player $A$ wins with 1990, player $B$ wins with 1. Determine winning
strategies for $n_0$.

6. Prove that there exists a convex 1990-gon with equal angles whose side lengths are
\[
1^2, 2^2, 3^2, \dots, 1990^2
\]
in some order.

\newpage

\section*{32nd International Mathematical Olympiad (1991)}

\subsection*{Day I}

1. Given a triangle $ABC$, let $I$ be the center of its incircle. The internal bisectors meet
the opposite sides at $A', B', C'$ respectively. Prove that
\[
\frac{1}{4} < \frac{AI \cdot BI \cdot CI}{AA' \cdot BB' \cdot CC'} \le \frac{8}{27}.
\]

2. Let $n > 6$ be an integer and let $a_1,a_2,\dots,a_k$ be all natural numbers less than $n$
and relatively prime to $n$. If
\[
a_2 - a_1 = a_3 - a_2 = \dots = a_k - a_{k-1} > 0,
\]
prove that $n$ is either a prime or a power of $2$.

3. Let $S = \{1,2,3,\dots,280\}$. Find the smallest integer $n$ such that every $n$-element
subset of $S$ contains five pairwise relatively prime numbers.

\subsection*{Day II}

4. Suppose $G$ is a connected graph with $k$ edges. Prove that the edges can be labeled
$1,2,\dots,k$ such that at each vertex incident with two or more edges, the greatest common
divisor of the labels is $1$.

5. Let $ABC$ be a triangle and $P$ an interior point. Show that at least one of the angles
\[
\angle PAB,\ \angle PBC,\ \angle PCA
\]
is less than or equal to $30^\circ$.

6. For any real number $a > 1$, construct a bounded infinite sequence
$x_0,x_1,x_2,\dots$ such that
\[
|x_i - x_j|\,|i - j|^a \ge 1
\]
for all distinct nonnegative integers $i,j$.

\end{document}
